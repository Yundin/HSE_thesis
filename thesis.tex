%!TEX TS-program = xelatex
% !TEX encoding = UTF-8 Unicode
% !TEX spellcheck = ru-RU
% !BIB program = biber

\documentclass[a4paper,14pt]{extreport} % ext for 14 font

\usepackage{etoolbox}
\newbool{comicsans}
\booltrue{comicsans}

%!TEX root = thesis.tex

\usepackage[english,russian]{babel}	% локализация и переносы
\usepackage{fontspec}
\ifbool{comicsans}{
	\setmainfont{Comic Sans MS}
}{
	\setmainfont{Times New Roman}
}
%\usepackage{tempora} % font for not xelatex

%\tolerance=1
%\emergencystretch=\maxdimen
%\hyphenpenalty=10000
%\hbadness=10000
\hyphenchar\font=-1
\sloppy

\usepackage{graphicx}
\usepackage{geometry}
	\geometry{left=2cm}
	\geometry{right=1cm}
	\geometry{top=2cm}
	\geometry{bottom=2cm}

\usepackage{setspace}
	\onehalfspacing

\setlength{\parindent}{1.25cm} % paragraph indent
\usepackage{indentfirst}
\setlength{\parskip}{0cm} % paragraph skip
\usepackage{titlesec}
\titlespacing*{\section}{\parindent}{*1}{*1}
\titlespacing*{\subsection}{\parindent}{*1}{*1}
\titlespacing*{\subsubsection}{\parindent}{*1}{*1}

\usepackage{multicol}
\usepackage{multirow}
\usepackage{tabularx}
\usepackage{makecell}

\newcommand{\titlefont}{\fontsize{18}{21.6}\bfseries\hyphenchar\font=-1}
\usepackage{titlesec}
\titleformat{\section}[block]
{\titlefont} {\thesection}{1em}{}
\titleformat{\subsection}[block]
{\titlefont} {\thesubsection}{1em}{}
\titleformat{\subsubsection}[block]
{\titlefont} {\thesubsubsection}{1em}{}
\newcommand{\centertitle}[1]{
	\setlength\parskip{0pt}
	\begin{center}
		\setlength\parskip{1ex}
		{\titlefont \uppercase{#1}}
	\end{center}
}
\newcommand{\centertitletoc}[1]{
	\setlength\parskip{0pt}
	\begin{center}
		{\titlefont \uppercase{#1}}
		\phantomsection
		\addcontentsline{toc}{section}{#1}
	\end{center}
}

\usepackage{caption}
\DeclareCaptionFont{captionsize}{\fontsize{13}{15.6}\selectfont}
\captionsetup{
	font=captionsize,
	justification=centering,
	labelsep=period,
	figurewithin=none,
	tablewithin=none,
	margin=1cm
}
\captionsetup[table]{
	justification=RaggedLeft,
	singlelinecheck=false,
	labelsep=newline,
	margin=0cm,
	skip=4pt
}
\usepackage{float}

\usepackage{tocloft} % toc style
\setlength{\cftsecnumwidth}{0pt}
\renewcommand{\cftsecaftersnumb}{\hspace{1.5em}}

\newcommand{\ris}[1]{(рис.~#1)}

\usepackage[
	backend=biber,
	sorting=none,
	style=gost-numeric
]{biblatex}
\addbibresource{mendeley.bib}

\usepackage{hyperref}
\hypersetup{				% Гиперссылки
	unicode=true,           % русские буквы в разделах PDF
	pdftitle={Разработка системы контроля и управления энергопотреблением элементов графического интерфейса на мобильных устройствах},   % Заголовок
	pdfauthor={Юндин Владислав},      % Автор
	colorlinks=true,       	% false: ссылки в рамках; true: цветные ссылки
	linkcolor=black,          % внутренние ссылки
	urlcolor=black,
	citecolor=black,        % на библиографию
}

\addto\captionsrussian{%
	\renewcommand{\contentsname}%
	{Оглавление}%
}

\usepackage{pdfpages}

\begin{document}
	\thispagestyle{empty}
	\newgeometry{left=2cm,right=2cm,top=1.25cm,bottom=1.25cm}
	\begin{center}
		\small
		ФЕДЕРАЛЬНОЕ ГОСУДАРСТВЕННОЕ АВТОНОМНОЕ ОБРАЗОВАТЕЛЬНОЕ \\ УЧРЕЖДЕНИЕ ВЫСШЕГО ОБРАЗОВАНИЯ \\ 
		«НАЦИОНАЛЬНЫЙ ИССЛЕДОВАТЕЛЬСКИЙ УНИВЕРСИТЕТ \\ 
		«ВЫСШАЯ ШКОЛА ЭКОНОМИКИ» \\
			
		\vspace{5mm}
		
		МОСКОВСКИЙ ИНСТИТУТ ЭЛЕКТРОНИКИ И МАТЕМАТИКИ \\
		им. А.Н. ТИХОНОВА
		
		\normalsize
		\vspace{12mm}
		
		Юндин Владислав Андреевич
		
		\vspace{12mm}
		
		\textbf{РАЗРАБОТКА СИСТЕМЫ КОНТРОЛЯ И УПРАВЛЕНИЯ ЭНЕРГОПОТРЕБЛЕНИЕМ ЭЛЕМЕНТОВ ГРАФИЧЕСКОГО ИНТЕРФЕЙСА НА МОБИЛЬНЫХ УСТРОЙСТВАХ}
		
		\vspace{8mm}
		
		Выпускная квалификационная работа по направлению подготовки \\ 
		09.03.01. Информатика и вычислительная техника \\
		студента образовательной программы \\
		<<Информатика и вычислительная техника>>
	\end{center}

	\vspace{10mm}

	\begin{multicols}{2}
		\noindent
		\textbf{Студент}
		
		\vspace{4mm}
		
		\noindent
		\begin{tabularx}{\linewidth}{Xc}
			& / В.А. Юндин \\
			\hline
		\end{tabularx}
		
		\columnbreak
		
		\noindent
		\textbf{Руководитель}
		
		\vspace{4mm}
		
		\noindent
		\begin{tabularx}{\linewidth}{c}
			должность, звание, А.Ю. Ролич \\
			\hline
		\end{tabularx}
	
		\vspace{8mm}
	
		\noindent
		\textbf{Консультант}
		
		\vspace{4mm}
		
		\noindent
		\begin{tabularx}{\linewidth}{c}
			должность, звание, И.О. Фамилия \\
			\hline
		\end{tabularx}
	
		\vspace{8mm}
		
		\noindent
		\textbf{Рецензент}
		
		\vspace{4mm}
		
		\noindent
		\begin{tabularx}{\linewidth}{c}
			должность, звание, И.О. Фамилия \\
			\hline
		\end{tabularx}
	\end{multicols}

	\vfill
	\begin{center}Москва, 2020\end{center}

	\restoregeometry
	\newpage
	
	\chapter{Chapter}
	
	\section{Section}
	
	\subsection{Subsection}
	
	\subsubsection{Subsubsection}
	
	\paragraph{Paragraph}
	
	\subparagraph{Subparagraph}
	
	
	Text Text Text Text Text Text Text Text Text Text Text Text Text Text Text Text Text Text Text Text 
	\begin{table}[h]
		\caption{ Энергетический базис деления ядра урана-235}
		\begin{tabularx}{\textwidth}{|X|c|c|c|}
			\hline
			\multicolumn{1}{|c|}{\multirow{2}{*}{Вид энергии}} & \multicolumn{2}{c|}{Энергия} & \multirow{2}{*}{\makecell[c]{Доля от общего количества\\ высвобождающейся энергии, \%}} \\ \cline{2-3}
			& Пдж  & МэВ &      \\ \hline
			Кинетическая энергия осколков деления              & \makecell[bc]{26,9} & 168 & 83,5 \\ \hline
			То же, нейтронов деления                           & 0,8  & 5   & 2,5  \\ \hline
			Энергия радиоактивного излучения продуктов деления & 2,9  & 18  & 9,0  \\ \hline
			Энергия нейтрино, испускаемых продуктами деления   & 1,6  & 10  & 5,0  \\ \hline
			Всего                                              & 32,2 & 201 & 100  \\ \hline
		\end{tabularx}
	\end{table}

	\captionof{figure}{Asd kaef Asd kaef Asd kaef Asd kaef Asd kaef Asd kaef Asd kaef Asd kaef Asd kaef Asd kaef Asd kaef }
	$\varphi$
\end{document}
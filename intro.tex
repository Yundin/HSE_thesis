%!TEX root = thesis.tex

\centertitletoc{Введение}

Современная жизнь немыслима без смартфонов и разных мобильных приложений, которые могут неэффективно потреблять ресурсы устройства. Значительные проблемы в оптимизации пользовательского опыта возникают в вопросах разряда батареи. Энергоэффективность приложений --- одна из важнейших проблем, с которой сталкиваются как разработчики, так и пользователи~\parencite{man2016experience, wasserman2010software}. Низкая энергоэффективность приложения ускоряет разрядку смартфона и может даже стать основанием для удаления приложения~\parencite{ickin2017users}. Данная проблема имеет популярность среди исследователей, и является предметом большого количества работ. Предложено множество способов снижения энергопотребления, но мне хотелось бы проверить эффективность метода, который основан на замене менее эффективных виджетов на экране более эффективными.

\subparagraph{Цель и задачи}
Конечной целью выпускной квалификационной работы является разработка системы контроля и управления энергопотреблением элементов графического интерфейса на устройствах под управлением операционной системы Android. Для её достижения необходимо решить следующие задачи:
\begin{itemize}
	\item Проанализировать существующие исследования по оптимизации энергопотребления;
	\item Определить инструменты для измерения энергопотребления Android-смартфонов;
	\item Измерить энергопотребление различных элементов пользовательского интерфейса;
	\item Создать библиотеку для мониторинга и управления энергопотреблением графических элементов пользовательского интерфейса
\end{itemize}

\subparagraph{Практическая значимость}
Решение данной проблемы в первую очередь представляет интерес для тех, кто занимается разработкой приложений для Android. Любой инженер, которому важно количество потребляемой приложением энергии, сможет встроить библиотеку в приложение для выявления наиболее затратных элементов интерфейса. Также, полученная библиотека составит список рекомендаций по снижению нагрузки на аккумулятор устройства. Благодаря созданной библиотеке, будет увеличено время автономной работы устройств пользователей.

\subparagraph{Программные средства}
Для разработки системы будет использован язык программирования Kotlin, библиотека автоматического тестирования пользовательского интерфейса Kaspresso, а также инструмент анализа информации энергопотребления Battery Historian.
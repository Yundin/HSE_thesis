%!TEX root = thesis.tex

\centertitletoc{ВВЕДЕНИЕ}

Современная жизнь немыслима без смартфонов и разных мобильных приложений, которые могут неэффективно потреблять ресурсы устройства. Значительные проблемы в оптимизации пользовательского опыта возникают в вопросах разряда батареи. Энергоэффективность приложений --- одна из важнейших проблем, с которой сталкиваются как разработчики, так и пользователи [1, 2]. Низкая энергоэффективность приложения ускоряет разрядку смартфона и может даже стать основанием для удаления приложения [3]. Данная проблема имеет популярность среди исследователей, и в настоящее время она является предметом большого количества работ. Предложено множество способов снизить энергопотребление, но мне хотелось бы проверить эффективность метода, который основан на замене менее эффективных View на экране более эффективными.

Конечной целью данного проекта является разработка системы контроля и управления энергопотреблением элементов графического интерфейса на устройствах под управлением операционной системы Android. Для достижения я отвечу на следующие вопросы:
\begin{itemize}
	\item Какие существуют исследования по оптимизации энергопотребления и каковы их недостатки?
	\item Какие имеются инструменты для измерения энергопотребления Android-смартфонов?
	\item Сколько энергии потребляют различные графические элементы пользовательского интерфейса?
	\item Как спроектировать библиотеку для мониторинга и управления энергопотреблением графических элементов пользовательского интерфейса?
\end{itemize}

Решение проблемы может представлять интерес для тех, кто занимается разработкой приложений для Android. Любой инженер, которому важно количество потребляемой приложением энергии, сможет встроить библиотеку в приложение для выявления наиболее затратных элементов интерфейса. Также, полученная библиотека составит список рекомендаций по снижению нагрузки на аккумулятор устройства.

Для разработки системы будет использован язык программирования Kotlin, библиотека автоматического тестирования пользовательского интерфейса Kaspresso, а также инструмент анализа информации энергопотребления Battery Historian.

Введение представляет собой краткий реферат ВКР. Введение должно содержать 

основные результаты анализа существующих технических решений объекта разработки, 

актуальность, (почему нужно это исследовать)

новизну (не обязательно), (почему моё решение нужно)

цели и задачи работы, 

инструментальные и программные средства, используемые для выполнения ВКР и 

планируемые результаты.

Объем введения не должен превышать 10 процентов объёма ВКР (без учёта приложений)

\textbf{\Huge TODO}